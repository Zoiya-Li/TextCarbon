\section{Introduction}
\label{Introduction}

The global commitment to achieving net-zero emissions and limiting anthropogenic warming to 1.5°C, as established in the Paris Agreement \cite{rogelj2016paris} and reinforced in subsequent COP summits \cite{tollefson2021cop26}, has positioned carbon emission forecasting as a critical instrument in navigating the transformation of global energy and economic systems. Recent IPCC reports \cite{ipcc2022climate} emphasize that accurate emission trajectory predictions are no longer optional but essential for evidence-based climate policy formulation. Such forecasts directly inform the calibration of Nationally Determined Contributions (NDCs) \cite{vandyck2018improving}, guide investments exceeding 4 trillion dollars annually in decarbonization technologies \cite{iea2023worldenergy}, and strengthen societal resilience against unavoidable climate impacts \cite{hallegatte2016shock}. As emissions continue to rise despite mitigation efforts \cite{friedlingstein2022global}, enhancing the scientific rigor of carbon emission forecasting methodologies has become a paramount challenge for the research community and a prerequisite for effective climate action.

The field of carbon emission forecasting has evolved significantly over the past decade, progressing from traditional statistical approaches to sophisticated deep learning architectures. While classical methods like ARIMA variants and error correction models \cite{li2017forecasting} established the foundation for temporal modeling, recent years have witnessed revolutionary advances in time series forecasting paradigms. Transformer-based architectures have emerged as particularly powerful, with models like Autoformer \cite{wu2021autoformer}, FEDformer \cite{zhou2022fedformer}, and the recent iTransformer \cite{liu2023itransformer} demonstrating remarkable capabilities in capturing long-range dependencies. Concurrently, a growing body of work has proposed innovative architectural paradigms. For instance, TimesNet \cite{zhou2022timesnet} reformulates time series as 2D variations, while TimeMixer \cite{wang2023timemixer} utilizes decomposable multiscale mixing for more efficient representation learning. These approaches have significantly pushed performance boundaries further. However, recent studies have begun to critically reassess these advances, questioning whether transformers are universally optimal in all forecasting contexts \cite{zeng2023transformers}. In specific forecast scenarios, well-designed linear models can sometimes outperform complex architectures \cite{li2023revisiting}. Additionally, the evolving landscape has also seen the emergence of hybrid approaches that integrate multimodal data sources \cite{zhang2023crossformer}, although few have effectively incorporated textual information describing real-world events that impact emission patterns.

Despite these technological advancements, current forecasting models face fundamental limitations when confronted with non-ergodic, event-driven dynamics that characterize real-world carbon emissions \cite{peters2022climate}. These limitations manifest as three interconnected challenges: First, a significant \textbf{data integration gap} exists. Despite the abundance of emission measurements \cite{crippa2021edgar} and climate policy databases \cite{nascimento2022global}, there is still a scarcity of datasets that systematically align high-frequency carbon emission time series with structured textual information about pivotal socio-economic, policy, and environmental events \cite{lamb2021climate}. This absence fundamentally impedes empirically grounded research on event-emission interactions. Second, researchers face a profound \textbf{challenge in causal attribution} \cite{gasser2020understanding}, systematically identifying diverse event types, quantifying their often non-linear impacts on emission trajectories, and confidently attributing observed anomalies to specific causal events remain formidable tasks. This results in an incomplete understanding of the true drivers underpinning emission dynamics \cite{liu2022drivers}. Third, a critical \textbf{limitation in predictive capability} emerges \cite{hausfather2020evaluating}. Existing models, architecturally ill-equipped to incorporate event-specific information and lacking integrated training data, struggle to generate forecasts sensitive to future events or policy interventions, severely limiting their utility for proactive climate governance \cite{grant2020cost}.

This research proposes a fundamental reorientation in carbon emission forecasting methodology. We argue that for predictive models to achieve genuine policy relevance in an era characterized by rapid transitions and disruptions \cite{geels2017sociotechnical}, the explicit incorporation of \textbf{event-driven dynamics} must become a central focus rather than a peripheral consideration. To address the aforementioned challenges, we introduce the \textbf{Text-Carbon Dataset}, a novel, extensive, and publicly accessible multi-modal resource specifically designed to enable event-centric carbon emission research. This dataset uniquely synthesizes daily, sectorally-disaggregated carbon emission time series across multiple administrative regions with AI-corroborated textual descriptions of relevant socio-economic, policy, and environmental events. Our methodological framework integrates advanced change-point detection algorithms \cite{truong2020selective} with Large Language Model (LLM)-based techniques for automated, context-aware retrieval and alignment of event narratives \cite{zhao2023survey}. The specific contributions of this work are:

\begin{itemize}
    \item We conceptualize, construct, and openly disseminate the Text-Carbon Dataset to address the critical \textit{data integration gap}, enabling researchers to explore the complex relationships between documented events and observed emission patterns.
    \item We propose TimeText, a novel \textbf{time series forecasting model that explicitly integrates textual event information} to directly address the \textit{limitation in predictive capability}. This model generates more accurate and interpretable carbon emission predictions in response to non-cyclical event-driven shocks.
    \item We comprehensively evaluate our approach using daily carbon emission data across aviation, ground transportation, industry, power generation, and residential sectors in 13 Chinese provinces, demonstrating consistent performance improvements over state-of-the-art baselines, particularly during periods of significant event-driven disruptions.
\end{itemize}

The remainder of this paper is organized as follows: Section \ref{sec:Related_Work} undertakes a critical review of the extant literature in environmental modeling and time series forecasting. Section \ref{sec:Data} offers a comprehensive exposition of the Text-Carbon Dataset, detailing its rigorous construction pipeline. Section \ref{sec:Methodology} delineates our proposed TimeText framework and Section \ref{sec:Discussion} presents the empirical findings derived from extensive experiments. Finally, Section \ref{sec:Conclusion} recapitulates the core contributions of the research, acknowledges its limitations, and proffers avenues for future inquiry.