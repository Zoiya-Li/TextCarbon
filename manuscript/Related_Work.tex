\section{Related Work}
\label{sec:Related_Work}

The intersection of carbon emission forecasting, time series analysis, and event-driven modeling spans multiple disciplines, requiring an interdisciplinary approach that draws from both environmental science and computer science methodologies.   To provide a comprehensive context for our work, we organize this review into three complementary domains: (1) carbon emission forecasting techniques in environmental science, which examines the evolution of methodologies specifically designed for emission prediction;   (2) time series forecasting models in machine learning, which explores the algorithmic advances in temporal data analysis;   and (3) multimodal fusion approaches for environmental modeling, which investigates techniques that integrate heterogeneous data sources to enhance predictive performance.

\subsection{Carbon Emission Forecasting in Environmental Science}
\label{subsec:carbon_forecasting}

Carbon emission forecasting in environmental science has evolved from simple statistical extrapolations to sophisticated integrated assessment models that incorporate socioeconomic factors, technological developments, and policy scenarios. Early approaches relied primarily on trend extrapolation and growth curve models \cite{schmalensee1998world}, which projected historical emission patterns into the future under assumptions of business-as-usual scenarios. These were succeeded by more sophisticated econometric models that established statistical relationships between emissions and economic indicators such as GDP, energy prices, and population growth \cite{holtz2018carbon}. The IPAT framework (Impact = Population × Affluence × Technology) and its stochastic extension STIRPAT \cite{york2003stirpat} formalized these relationships, enabling researchers to decompose emission drivers quantitatively. Contemporary approaches have gravitated toward integrated assessment models (IAMs) such as IMAGE \cite{stehfest2014integrated}, GCAM \cite{calvin2019gcam}, and REMIND \cite{luderer2015description}, which simulate complex interactions between economic, technological, and environmental systems to generate emission scenarios under various policy interventions. These models have been instrumental in informing international climate negotiations and national policy formulation \cite{riahi2017shared}. Recent refinements have incorporated machine learning techniques to improve parameter estimation and uncertainty quantification \cite{burandt2018big}, while also developing specialized models for specific sectors such as energy \cite{creutzig2022demand}, transportation \cite{yin2015china}, and industry \cite{karali2020carbon}. Despite these advances, current environmental science approaches to emission forecasting remain limited in their ability to capture and respond to abrupt, event-driven changes in emission patterns, particularly at high temporal resolutions (daily or weekly), as they typically operate on annual or decadal timescales and prioritize long-term structural relationships over short-term fluctuations.

\subsection{Time Series Forecasting Models in Machine Learning}
\label{subsec:time_series_forecasting}

Time series forecasting in machine learning has witnessed remarkable methodological innovations, progressing from classical statistical models to sophisticated deep learning architectures. Traditional approaches such as ARIMA, exponential smoothing, and state space models \cite{hyndman2018forecasting} established the foundation for time series analysis by decomposing temporal data into trend, seasonality, and residual components. The emergence of machine learning introduced more flexible models capable of capturing non-linear patterns, with Support Vector Regression \cite{smola2004tutorial}, Random Forests \cite{breiman2001random}, and Gradient Boosting \cite{friedman2001greedy} demonstrating superior performance in various forecasting competitions. The deep learning revolution subsequently transformed the field, beginning with Recurrent Neural Networks (RNNs) and their variants such as Long Short-Term Memory (LSTM) \cite{hochreiter1997long} and Gated Recurrent Units (GRU) \cite{cho2014learning}, which addressed the vanishing gradient problem and enabled more effective modeling of long-range dependencies. Attention mechanisms \cite{vaswani2017attention} further enhanced these capabilities, leading to the development of Transformer-based architectures that have achieved state-of-the-art results across numerous benchmarks. Recent innovations include Temporal Fusion Transformers \cite{lim2021temporal}, which combine recurrent layers with self-attention for interpretable multi-horizon forecasting; N-BEATS \cite{oreshkin2019n}, which employs deep neural networks with backward and forward residual links; and specialized architectures like Autoformer \cite{wu2021autoformer}, FEDformer \cite{zhou2022fedformer}, and TimesNet \cite{zhou2022timesnet}, which incorporate domain-specific inductive biases for time series data. Despite these advances, most machine learning approaches to time series forecasting remain predominantly focused on numerical data and struggle to incorporate external, non-structured information such as textual descriptions of events that might significantly impact the time series, limiting their effectiveness in domains where such contextual information is crucial for accurate prediction.

\subsection{Multimodal Fusion Approaches for Environmental Modeling}
\label{subsec:multimodal_fusion}

Multimodal fusion approaches for environmental modeling have gained prominence as researchers recognize the value of integrating diverse data sources to enhance predictive accuracy and explanatory power. Early fusion strategies employed simple concatenation or averaging of features derived from different modalities, such as combining satellite imagery with ground-based measurements for land use classification \cite{joshi2016review}. More sophisticated approaches have leveraged canonical correlation analysis \cite{hardoon2004canonical} and multiple kernel learning \cite{gonen2011multiple} to identify and exploit cross-modal correlations while respecting the unique statistical properties of each data source. The advent of deep learning has enabled end-to-end trainable fusion architectures, with modality-specific encoders that project heterogeneous inputs into a shared latent space where they can be effectively combined \cite{baltruvsaitis2018multimodal}. In environmental science specifically, researchers have developed fusion frameworks that integrate numerical time series with satellite imagery \cite{reichstein2019deep}, meteorological data with social media signals \cite{wang2019social}, and sensor networks with physics-based models \cite{reichstein2019deep}. Recent work has explored attention-based fusion mechanisms that dynamically weight different modalities based on their relevance to the prediction task \cite{xu2018multimodal}, as well as graph neural networks that can model complex spatial and temporal dependencies across heterogeneous data sources \cite{wu2020connecting}. Despite these advances, existing multimodal fusion approaches in environmental modeling have largely overlooked the integration of structured time series data with unstructured textual information about events and policies, particularly in the context of carbon emission forecasting where such integration could significantly enhance predictive performance during periods of policy-induced or event-driven emission changes.

In summary, while significant progress has been made across all three domains—carbon emission forecasting in environmental science, time series forecasting in machine learning, and multimodal fusion for environmental modeling—a critical gap remains in developing approaches that can effectively integrate high-frequency carbon emission time series with textual information about relevant events and policies. Our work addresses this gap by proposing a novel framework that combines advanced time series modeling techniques with text embedding methods to capture the impact of documented events on emission patterns, thereby enhancing both the accuracy and interpretability of carbon emission forecasts.
